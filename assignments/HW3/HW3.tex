\documentclass[11pt]{report}
\usepackage{amsmath}
\usepackage[notes]{ctmmath-v2}
\usepackage[left=2cm,top=2cm,right=2cm,bottom=1.5cm]{geometry} %changes margins
\usepackage{graphicx}
\usepackage{amsmath}
\usepackage{epstopdf}
\usepackage{enumerate}
\pagestyle{empty}
\usepackage{booktabs}
\usepackage{enumitem}
\usepackage{tikz}
\usepackage{hyperref}
\usepackage[sectionbib]{natbib}
\renewcommand{\bibsection}{\section*{References}}
\setlength\parindent{0pt}
 

\begin{document}



\begin{center}
\begin{Large}\textbf{ENVR 453: Groundwater Hydrology \\ Assignment 3: Groundwater Equation} \end{Large}
\\ \vspace{0.1cm} Timothy M. Weigand \vspace{0.1cm}\\ 
Due: October 12th, 2023
\end{center}

\vspace{0.2cm}


\begin{itemize}
\item {\bf Problem 1 (15 points) } \\
A community is in need of an additional water source and is considering drilling a new well in the local aquifer. Farmers in the area rely on the aquifer for irrigation and if the average hydraulic head of the aquifer drops below 10\% of the current value, the proposed project will be rejected because this would require the farmers to modify their existing wells and pumps.  

The confined aquifer has a length of 1,000 m and is bound on the left by a lake that maintains a constant head of 100 m and by a steam on the right that also has a steady head of 85 m. 
Due to the symmetry of the subsurface, a 1-dimensional model should be sufficient to accurately model the system. 


\begin{enumerate}[label=(\roman*)]
    \item Assuming that the hydraulic conductivity ($K$) of the system is 10 m/day and the aquifer has a porosity of 0.3, determine the velocity of the water in the system by only using boundary information. 
    \item How long does it take water to travel the entire length of the aquifer? 
    \item Derive the governing equations for this aquifer and explicitly state all assumptions made. You can start with the 1-dimensional groundwater equations which is defined as
    \beq
    S_s \pdt{h} - \pd{}{x} \lp K \pd{h}{x} \rp + q = 0
    \eeq
    \item For your governing equation, develop a numerical approximation using the finite difference approach using a total of 5 nodes and assume there are no sources or sinks. Formulate your approach such that it is in the form of 
    \beq
    \ten A \vec h  = \vec b
    \eeq
    \item Implement your finite difference approximation in Python where the boundary conditions, number of nodes, and length of the domain can be specified. 
    \item A key component of the modeling process is model verification. Prove that your numerical model is accurately solving the governing equation.
    \item Using you verified model and 251 nodes for your finite difference approximation, determine the maximum pumping rate given the established criteria for the proposed project. Implementation of a well in a finite difference code is non-trivial. To lessen the burden, we will assume that the well is located at x = 400 m (node 100). The units of $q$ are 1/day and to convert this value to a flow rate we need to multiply by a volume. We will assume a volume of 1,000 m$^3$. Perform the same analysis at x = 800 m (node 200). Are the maximum allowable pumping rates the same? Why or why not? 

\end{enumerate}

\end{itemize}


\end{document}
